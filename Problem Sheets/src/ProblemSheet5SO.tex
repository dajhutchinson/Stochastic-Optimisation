\documentclass[11pt,a4paper]{article}

\usepackage[margin=1in, paperwidth=8.3in, paperheight=11.7in]{geometry}
\usepackage{amsmath,amsfonts,fancyhdr,bbm,changepage}
\usepackage[section,nohyphen]{DomH}
\headertitle{Stochastic Optimisation - Problem Sheet 5}

\begin{document}

% \questionsfalse
% \answersfalse

\title{Stochastic Optimisation - Problem Sheet 5}
\author{Dom Hutchinson}
\date{\today}
\maketitle

\begin{question}{3)}
  Consider a queueing system with room for $n$ customers which operates over $N$ time periods (denoted by $0,\dots,N-1$). Regarding this system, we assume the following
  \begin{itemize}
    \item Only one customer can arrive during a period. A new customer arrives during a period with probability $p$.
    \item Customer arrivals in different periods are independent.
    \item A new customer is either allowed to join the queue or be rejected. Rejected customers depart without attempting to re-enter.
    \item The service of a customer can start/end at the beginning/end of a period.
    \item A customer in service at the beginning of a period terminates the service at the end of the period with probability $q$, with $q<p$. This is independent of the number of periods the customer has been in service and of the number of customers in the system.
    \item The cost of rejecting a customer is $C$. The cost of maintaining $i$ customers in the queue over a single period is $ci$. The cost of admitting a new customer is $c$.
  \end{itemize}
  The problem is to decide, at each time period, whether to accept or reject a new customer so as the expected total cost is minimal. Each decision should be based on the current number of customers in the queue.
\end{question}

\begin{question}{3) (a)}
  Formulate the problem of minimizing the total expected cost as a \textit{Markov Decision Problem}. Derive the optimality equations for the formulated decision problem.
\end{question}

\begin{answer}{3) (a)}
  Let $X_t$ be the system state at the start of time-period $t$ and $Y_t$ be the action the agent takes at the start of time-period $t$.
  \par Let $W_t$ represent the number of people wishing to join the queue in time-period $t$ and $Z_t$ represent the number of customers who leave the queue in time-period $t$. From the question, $W_t,Z_t$ have the following distributions
  \[\begin{array}{rcl}
    \prob(W_t=w)&=&\begin{cases}
      p&w=1\\
      1-p&w=0
    \end{cases}\\
    \prob(Z_t=z)&=&\begin{cases}
      q&z=1\\
      1-q&z=0
    \end{cases}
  \end{array}\]
  \begin{itemize}

    \item \textit{Decision Epochs} - At the start of each period.

    \item \textit{Time-Horizon} - $T=\{0,\dots,N-1\}$.

    \item \textit{Agent Action}.
    \par At the start of each time-period, let the agent decide whether to allow customers to join the queue or not. This can be encoded into $Y_t$ as
    \[ Y_t=\begin{cases}
    1&\text{if accepting customers}\\
    0&\text{if \underline{not} accepting customers}
    \end{cases} \]
    As at most 1 customer may wish to join the queue in a single time-period, $Y_t$ is equivalent to the maximum amount of customer being allowed to join the queue in time-period $t$.

    \item \textit{Action-Space}.
    \par Given the specification of $Y_t$ the action-space is $A=\{0,1\}$.

    \item \textit{System States}.
    \par Knowing the number of customers currently in the queue is sufficient for the agent to determine which action to take. This knowledge can be encoded into $X_t$, by setting $X_t$ to be the number of people in the queue at the start of time-period $t$.
    \par This gives us the state-equation $X_{t+1}=X_t+Y_tW_t-Z_t$.

    \item \textit{State-Space}
    \par By the definition of $X_t$ and the fact that the maximum queue capacity is $n$, the state-space is $S=\{0,1,\dots,n\}$.

    \item \textit{Admissible Action-Space}.
    \par The agent is not allowed to let customers join the queue if the queue is already full.
    \[ A(s)=\begin{cases}\{0,1\}&\text{if }s\in[0,\dots,n-1]\\\{0\}&\text{if }s=n\end{cases} \]

    \item \textit{Transition Probabilities}.
    \par The definition of transition probabilities states
    \[ p_t(s'|s,a):=\prob^\pi(X_{t+1}=s'|X_t=s,Y_t=a) \]
    By considering the state-equation for our system, we can derive the following
    \[\begin{array}{rcl}
      p_t(s'|s,a)&=&\prob^\pi(X_{t+1}=s'|X_t=s,Y_t=a)\\
      &=&\prob^\pi(X_t+Y_tW_t-Z_t=s'|X_t=s,Y_t=a)\\
      &=&\prob(s+aW_t-Z_t=s')\\
    \end{array}\]
    To get a deterministic value for this expression, we consider the following cases involving $s',s$
    \begin{enumerate}

      \item $s=0$.
      \par The queue is empty so no-one can leave the queue (ie $\prob(Z_t=0)=1$).
      \[\begin{array}{rcl}
        p_t(s'|0,a)&=&\prob(aW_t=s')\\
                   &=& \begin{cases}
                         1&\text{if }a=0,s'=0\\
                         \prob(W_t=0)&\text{if }a=1,s'=0\\
                         \prob(W_t=1)&\text{if }a=1,s'=1\\
                         0&\text{otherwise}
                       \end{cases}\\
                   &=& \begin{cases}
                         1&\text{if }a=0,s'=0\\
                         1-p&\text{if }a=1,s'=0\\
                         p&\text{if }a=1,s'=1\\
                         0&\text{otherwise}
                       \end{cases}\\
      \end{array}\]

      \item $s=n$.
      \par The queue is full so no-one can join the queue (ie $\prob(W_t=0)=1$).
      \[\begin{array}{rcl}
        p_t(s'|n,a)&=&\prob(n-Z_t=s')\\
                   &=& \begin{cases}
                         \prob(Z_t=0)&\text{if }s'=n\\
                         \prob(Z_t=1)&\text{if }s'=n-1\\
                         0&\text{if }s'<n-1
                       \end{cases}\\
                   &=& \begin{cases}
                         1-q&\text{if }s'=n\\
                         q&\text{if }s'=n-1\\
                         0&\text{if }s'<n-1
                       \end{cases}\\
      \end{array}\]
      It is noteworthy that these results are independent of the action taken $a$.

      \item $s'=s,\ s\not\in\{0,n\}$.
      \par The queue has not changed size. Thus, either: no movements have occurred; or, one customer joined the queue and another customer left the queue. The second scenario is only possible if the agent allows customers to join the queue (ie $a=1$).
      \[\begin{array}{rcl}
        p_t(s|s,a)&=&\prob(s+aW_t-Z_t=s)\\
        &=&\prob(aW_t-Z_t=0)\\
        &=&\begin{cases}
             \prob(Z_t=0)&\text{if }a=0\\
             \prob(W_t=Z_t)&\text{if }a=1
           \end{cases}\\
        &=&\begin{cases}
            1-q&\text{if }a=0\\
            pq+(1-p)(1-q)&\text{if }a=1
          \end{cases}
      \end{array}\]

      \item $s'=s+1,\ s\not\in\{0,n\}$.
      \par The queue has become longer.
      \[\begin{array}{rcl}
        p_t(s+1|s,a)&=&\prob(s+aW_t-Z_t=s+1)\\
        &=&\begin{cases}
             \prob(-Z_t=1)&\text{if }a=0\\
             \prob(W_t-Z_t=1)&\text{if }a=1
           \end{cases}\\
        &=&\begin{cases}
             0&\text{if }a=0\\
             \prob(W_t=1)\prob(Z_t=0)&\text{if }a=1
           \end{cases}\\
        &=&\begin{cases}
             0&\text{if }a=0\\
             p(1-q)&\text{if }a=1
           \end{cases}
      \end{array}\]

      \item $s'=s-1,\ s\not\in\{0,n\}$.
      \par The queue has become shorter.
      \[\begin{array}{rcl}
        p_t(s-1|s,a)&=&\prob(s+aW_t-Z_t=s-1)\\
        &=&\prob(aW_t-Z_t=-1)\\
        &=&\begin{cases}
             \prob(Z_t=1)&\text{if }a=0\\
             \prob(W_t-Z_t=-1)&\text{if }a=1
           \end{cases}\\
        &=&\begin{cases}
             q&\text{if }a=0\\
             \prob(W_t=0,Z_t=1)&\text{if }a=1
           \end{cases}\\
        &=&\begin{cases}
             q&\text{if }a=0\\
             (1-p)q&\text{if }a=1
           \end{cases}
      \end{array}\]

      \item \textit{All other cases.}
      \par All other cases require either the number of people in the queue to become negative or to change by more than 1 in a single time-step, both of these are impossible so have 0 probability.
      \[ p_t(s'|0,a)=0 \]

    \end{enumerate}

    \item \textit{Immediate Costs}
    \par For costs in time-period $t$ we always have to pay $cX_t$ for the length of the queue. Additionally, if a customer is rejected $C$ is payed as well. Mathematically, a customer is rejected if $W_t=1$ and $Y_t=0$. We can summarise the total cost $G_t$ incurred in time-period $t$ as
    \[ G_t:=g_t(W_t,X_t,Y_t)=cX_t+\indexed\{W_t=1,Y_t=0\}\cdot C \]
    This assumes that $Y_t=0$ if the queue is already full.

    \item \textit{Equivalent Objective}
    \par The objective of this problem is to minimise expected total cost
    \[ \expect^\pi\left[\sum_{t=0}^{N-1}G_t\right]=\expect^\pi\left[\sum_{t=0}^{N-1}g_t(W_t,X_t,Y_t)\right] \]
    This expectations depends upon the number of customers wishing to join the queue $Y_0,\dots,Y_{N-1}$ which is independent of a policy $\pi$ so does not fit within the framework a \textit{Markov Decision Problem}. Thus I shall transform the expect total cost.
    \[\begin{array}{rcl}
      \expect^\pi\left[\sum_{t=0}^{N-1}G_t\right]&=&\sum_{t=0}^{N-1}\expect^\pi[G_t]\\
      &=&\sum_{t=0}^{N-1}\expect^\pi\big[\expect^\pi[G_t|X_t,Y_t]\big]\text{ by Tower Property}\\
      &=&\sum_{t=0}^{N-1}\expect^\pi\big[\expect^\pi[g_t(W_t,X_t,Y_t)|X_t,Y_t]\big]\\
    \end{array}\]
    Define reward functions $r_t(s,a)$ and $r_N(s)$
    \[\begin{array}{rcl}
      r_N(s)&=&0\\
      r_t(s,a)&=&-\expect^\pi[g_t(W_t,X_t,Y_t)|X_t=s,Y_t=a]\\
      &=&-\expect^\pi[g_t(W_t,s,a)|X_t=s,Y_t=a]\\
      &=&-\expect^\pi[g_t(W_t,s,a)]\\
      &=&-p\cdot g_t(1,s,a)-(1-p)\cdot g_t(0,s,a)\\
      &=&-p(cs+\indexed\{a=0\}C)-(1-p)cs\\
      &=&-cs-\indexed\{a=0\}\cdot pC
    \end{array}\]
    Using these reward functions the total expected reward can be rephrased
    \begin{equation} \label{eq_objective3}
      -\expect^\pi\left[r_N(X_n)+\sum_{t=0}^{N-1}r_t(X_t,Y_t)\right]
    \end{equation}
    The equivalent objective is to find a policy $\pi\in HR(T)$ which maximises $(\ref{eq_objective3})$.
  \end{itemize}
\end{answer}

\begin{question}{3) (b)}
  Solve the formulated Markov decision problem for the following case
  \begin{itemize}
    \item $N=2,n=3$.
    \item $p=\frac12,q=\frac14$.
    \item $C=2,c=1$.
  \end{itemize}
\end{question}

\begin{answer}{3) (b)}
  From the markov decision problem formulated in \texttt{3) (a)} and the given conditions we can state the following properties of the system
  \[\begin{array}{rcl}
    T&=&\{0,1\}\\
    S&=&\{0,1,2,3\}\\
    A&=&\{0,1\}\\
    A(s)&=&\begin{cases}
             \{0,1\}&\text{if }s\in\{0,1,2\}\\
             \{0\}&\text{if }s=3
           \end{cases}\\
  \end{array}\]
  The transition probabilities $p_t(s'|s,a)$ are defined in the tables below, separated by what action $a$ is taken.
  \begin{adjustwidth}{-2cm}{}
    \begin{tabular}{rl}
      $p_t(s'|s,0)=$&\begin{tabular}{c|cccc}
      s\textbackslash s'&0&1&2&3\\\hline
      0&1&0&0&0\\
      1&q&1-q&0&0\\
      2&0&q&1-q&0\\
      3&0&0&q&1-q
      \end{tabular}$=$
      \begin{tabular}{c|cccc}
        s\textbackslash s'&0&1&2&3\\\hline
        0&1&0&0&0\\
        1&1/4&3/4&0&0\\
        2&0&1/4&3/4&0\\
        3&0&0&1/4&3/4
      \end{tabular}\\\\
      $p_t(s'|s,1)=$&\begin{tabular}{c|cccc}
      s\textbackslash s'&0&1&2&3\\\hline
      0&1-p&p&0&0\\
      1&q(1-p)&pq+(1-p)(1-q)&p(1-q)&0\\
      2&0&q(1-p)&pq+(1-p)(1-q)&p(1-q)\\
      3&0&0&q&1-q
      \end{tabular}$=$
      \begin{tabular}{c|cccc}
        s\textbackslash s'&0&1&2&3\\\hline
        0&1/2&1/2&0&0\\
        1&1/4&1/8&3/8&0\\
        2&0&1/4&1/8&3/8\\
        3&0&0&1/4&3/4
      \end{tabular}
    \end{tabular}
  \end{adjustwidth}
  The terminal cost value is $r_2(s)=0$. The cost function values $r_t(s,a)$ are given in the table below and are the same $\forall\ t\in\{0,1\}$.
  \begin{center}
    $r_t(s,a)=$\begin{tabular}{c|cc}
      s\textbackslash a&0&1\\\hline
      0&$-pC$&$0$\\
      1&$-c-pC$&$-c$\\
      2&$-2c-pC$&$-2c$\\
      3&$-3c-pC$&$-3c$\\
    \end{tabular}
    $=$
    \begin{tabular}{c|cc}
      s\textbackslash a&0&1\\\hline
      0&-1&0\\
      1&-2&-1\\
      2&-3&-2\\
      3&-4&-3
    \end{tabular}
  \end{center}
  To find the optimal policy $\pi^*$ we use the \textit{dynamic programming algorithm} which is defined as
  \[\begin{array}{rrl}
    w_t^*(s,a)&:=&r_t(s,a)+\sum_{s'\in S}u_{t+1}^*(s')p_t(s'|s,a)\\
    u_t^*(s)&=&\max_{a\in A(s)}(w_t^*)\\
    d_t^*(s)&=&\argmax_{a\in A(s)}(w_t^*)
  \end{array}\]
  where $u_2^*(s):=r_2(s)=0\ \forall\ s\in S$. Specifically, we need to determine $u_t^*(s),\ d_t^*(s)$ for all states $s$ in each time-period $t\in\{1,0\}$.
  \begin{itemize}
    \item \textit{Time-Period} $t=1$.
    \par In this time-period
    \[\begin{array}{rcl}
      w_1^*(s,a)&=&r_1(s,a)+\sum_{s'\in\{0,1,2,3\}}u_2^*(s)p_t(s'|s,a)\\
      &=&r_1(s,a)\text{ since }u_2^*(s)=0\ \forall\ s\in S
    \end{array}\]
    This gives the following table of values for $w_1^*(s,a)$
    \begin{center}
      $w_1^*(s,a)=$
      \begin{tabular}{c|cc}
        s\textbackslash a&0&1\\\hline
        0&-1&0\\
        1&-2&-1\\
        2&-3&-2\\
        3&-4&-3
      \end{tabular}
    \end{center}
    This table shows that taking action $a=1$ produces the greatest expected reward in all states $s$. This is summarised in the following table for $u_1^*(s),d_1^*(s)$
    \begin{center}
      \begin{tabular}{c|c|c}
        $s$&$u_1^*(s)$&$d_1^*(s)$\\\hline
        0&0&1\\
        1&-1&1\\
        2&-2&1\\
        3&-3&1
      \end{tabular}
    \end{center}

    \item \textit{Time-Period} $t=0$.
    \par In this time-period
    \[\begin{array}{rcl}
      w_0^*(s,a)&=&r_0(s,a)+\sum_{s'\in\{0,1,2,3\}}u_1^*(s)p_t(s'|s,a)\\
      &=&r_0(s,a)-p_t(1|s,a)-2\cdot p_t(2|s,a)-3\cdot p_t(3|s,a)
    \end{array}\]
    This gives the following table of values for $w_0^*(s,a)$
    \begin{center}
      $w_1^*(s,a)=$
      \begin{tabular}{c|cc}
        s\textbackslash a&0&1\\\hline
        0&-1+0+0+0&0-1/2+0+0\\
        1&-2-3/4+0+0&-1-1/8-6/8+0+0\\
        2&-3-1/4-6/4+0&-2-1/4-2/8-9/8\\
        3&-4+0-2/4-9/4&-3+0-2/4-9/4
      \end{tabular}
      =
      \begin{tabular}{c|cc}
        s\textbackslash a&0&1\\\hline
        0&-1&-1/2\\
        1&-11/4&-15/8\\
        2&-19/4&-29/8\\
        3&-27/4&-23/4
      \end{tabular}
    \end{center}
    Again, this table shows that taking action $a=1$ produces the greatest expected reward in all states $s$. This is summarised in the following table for $u_1^*(s),d_1^*(s)$
    \begin{center}
      \begin{tabular}{c|c|c}
        $s$&$u_0^*(s)$&$d_0^*(s)$\\\hline
        0&-1/2&1\\
        1&-15/8&1\\
        2&-29/8&1\\
        3&-23/4&1
      \end{tabular}
    \end{center}
  \end{itemize}
  The optimal policy is
  \[ \pi^*=(d_0^*(s),d_1^*(s))=(1,1)\ \forall\ s \]
  The optimal value function is
  \[ u_0^*(s)=\begin{cases}
                -1/2&\text{if }s=0\\
                -15/8&\text{if }s=1\\
                -29/8&\text{if }s=2\\
                -23/4&\text{if }s=3
              \end{cases} \]
\end{answer}

\newpage
\begin{question}{4)}
  Consider the following machine maintenance problem.
  \par A machine is operated over time-periods $0,\dots,N-1$. The machine can be in one of states $1,\dots,M$. For $s\in\{1,\dots,M-1\}$, we assume that the condition of the machine is better in state $s$ than in state $s+1$ (ie 1 is best condition, $M$ is worst condition).
  \par At the start of each period, the state of the machine is known and one of the following actions is taken
  \begin{itemize}
    \item The machine is allowed to operate in the current state for one more period.
    \item The machine is repaired: If the machine is in state $s\in\{2,\dots,M\}$ at the beginning of the time-period, then its state can immediately be restored to (a better) state in $s'\in\{1,\dots,s-1\}$ at cost $c_r(s,s')$.
  \end{itemize}
  The cost of operating the machine in state $s\in\{1,\dots,M\}$ during a time-period is $c_o(s)$. (If the machine is in state $s$ after the action taken at the beginning of a time-period, cost $c_0(s)$ is incurred during that period). We assume worst states cost more
  \[ c_o(1)<\dots<c_0(M) \]
  During each time-period, the state of the machine can become worse or it may stay unchanged. The machine changes its state randomly:
  \begin{itemize}
    \item If $s\in\{1,\dots,M-1\}$ is the state after the action taken at the beginning of the time-period, the machine will be in state $s'\in\{s,\dots,M\}$ at the end of this period with probability $p(s'|s)$ where $p(s'|s)\geq0$ and $\sum_{s'=s}^Mp(s'|s)=1$.
    \item If the machine is in state $M$ after the action taken at the beginning of the time-period, the machine will remain in state $M$ at the end of this period with probability one.
  \end{itemize}
  \par The problem is to decide, at the beginning of each time-period, what action to take so as to expect total cost of operating the machine over period $0,\dots,N-1$ is minimal. The decision should be based on the state of the machine.
\end{question}

\begin{question}{4) (a)}
  Formulate the described machine maintenance problem as a finite horizon Markov decision problem. More precisely, state the state-space $S$, action-space $A$, time-horizon $T$, transition probabilities $p_t(s'|s,a)$ and immediate rewards $r_t(s,a)$.
\end{question}

\begin{answer}{4) (a)}
  Let $X_t$ be the system state at the start of time-period $X_t$ and $Y_t$ be the action the agent takes at the start of time-period $Y_t$.
  \begin{itemize}
    \item \textit{Decision Epochs} - Start of each time-period.
    \item \textit{Time-Horizon} - $T=\{0,\dots,N-1\}$.
    \item \textit{Agent Actions}.
    \par At the start of each turn the agent can either repair the machine to a specified state $s\in\{1,\dots,M-1\}$ or not repair. This can be encoded into $Y_t$ as
    \[
      Y_t=\begin{cases}
            0&\text{if machine is \underline{not} repaired}\\
            s&\text{if machine is repaired to state }s
           \end{cases}
    \]

    \item \textit{Action-Space}.
    \par Given the specification of $Y_t$ the action-space is $A:=\{0,\dots,M-1\}$.

    \item \textit{System-States}.
    \par Knowing the state of the machine at the beginning of a time-period is sufficient for the agent to determine which action to take. This knowledge can be encoded into $X_t$ by setting $X_t$ to be the state of the machine at the start of period $t$.

    \item \textit{State-Space}.
    \par By the definition of $X_t$, the state-space is equivalent to the set of states the machine can take $S=\{1,\dots,M\}$.

    \item \textit{Admissible Action-Space}.
    \par The agent can always choose not to repair the machine, but if they choose to repair the machine they can only repair it to a better state. This gives admissible action-spaces
    \[ A(s)=\{0,\dots,s-1\},\ s\in S \]

    \item \textit{Immediate Costs}.
    \par In each time-period there is always a cost for running the machine $c_o(s)$, where $s$ is the state after the agent has taken their action. There is an additional cost $c_r(s,s')$ if the agent chooses to repair the machine from state $s$ to state $s'$ (ie if $Y_t=s'\neq0$).
    \par Let $Z_t$ be the state of the machine after the agent takes their action. If the agent does not repair (ie $Y_t=0$), then the state of the machine after the agent's action is the same as before (ie $Z_t=X_t$). However, if the agent chooses to repair the machine (ie $Y_t=s'\in A(X_t)\setminus\{0\}$) then the machine's state after the action will be whatever state the agent chose to repair the machine to (ie $Z_t=Y_t$). This gives the following definition for $Z_t$
    \[ Z_t=\begin{cases}
         X_t&\text{if }Y_t=0\\
         Y_t&\text{if }Y_t\in A(X_t)\setminus\{0\}
       \end{cases}\]
    We can define the total cost in time-period $t$ as
    \[\begin{array}{rrl}
      G_t&:=&c_o(Z_t)+\indexed\big\{Y_t\in A(X_t)\setminus\{0\}\big\}c_r(X_t,Z_t)\\
      &=&\begin{cases}
              c_o(X_t)&\text{if }Y_t=0\\
              c_o(Y_t)+c_r(X_t,Y_t)&\text{if }Y_t\in A(s)\setminus\{0\}
            \end{cases}
    \end{array}\]
    For clarity of dependency, define the function
    \[ g_t(X_t,Y_t):=\begin{cases}
                       c_o(X_t)&\text{if }Y_t=0\\
                       c_o(Y_t)+c_r(X_t,Y_t)&\text{if }Y_t\in A(s)\setminus\{0\}
                       \end{cases}\]

    \item \textit{Transition Probabilities}.
    \[\begin{array}{rrl}
      p_t(s'|s,a)&:=&\prob^\pi(X_{t+1}=s'|X_t=s,Y_t=a)\\
      &=&\begin{cases}
           p(s'|s)&\text{if }a=0\\
           p(s'|a)&\text{if }a\in A(s)\setminus\{0\}
         \end{cases}
    \end{array}\]
    where $p(\cdot|\cdot)$ are the probabilities of the machine degrading as defined in the question.

    \item \textit{Equivalent Objective}
    The objective of this problem is to minimises the total expected cost
    \[\begin{array}{rcl}
      \expect^\pi\left[\sum_{t=0}^{N-1}G_t\right]&=&\expect^\pi\left[\sum_{t=0}^{N-1}g_t(X_t,Y_t)\right]\\
      &=&\sum_{t=0}^{N-1}\expect^\pi[g_t(X_t,Y_t)]\\
      &=&\sum_{t=0}^{N-1}\expect^\pi\big[\expect^\pi[g_t(X_t,Y_t)]\big|X_t=s,Y_t=a\big]\text{ by Tower Property.}
    \end{array}\]
    Minimisation does not fit within the Markov decision problem framework, so I shall make this a maximisation problem. Consider the following reward functions
    \[\begin{array}{rrl}
      r_N(s)&:=&0\\
      r_t(s,a)&:=&-\expect^\pi[g_t(X_t,Y_t)|X_t=s,Y_t=a]\\
      &=&-\expect^\pi[g_t(s,a)]\\
      &=&\begin{cases}
              -c_o(s)&\text{if }a=0\\
              -c_o(s)-c_r(s,a)&\text{if }a\in\{1,\dots,s-1\}
            \end{cases}
    \end{array}\]
    Using these definitions, our objective can be restated as wishing to maximise the following
    \[ -\expect^\pi\left[r_N(s)+\sum_{t=0}^{N-1}r_t(s,a)\right] \]
  \end{itemize}
\end{answer}

\begin{question}{4) (b)}
  Find the optimal policy for the Markov decision problem formulated it \texttt{4) (a)} under the following conditions
  \begin{itemize}
    \item $M=3,N=2$.
    \item $p(1|1)=\frac12,\ p(2|1)=\frac14,\ p(3|1)=\frac14,\ p(2|2)=\frac14,\ p(3|2)=\frac34$.
    \item $c_r(2,1)=1,\ c_r(3,2)=2,\ c_r(3,1)=4$.
    \item $c_o(1)=1,\ c_o(2)=2,\ c_o(3)=5$.
  \end{itemize}
\end{question}

\begin{answer}{4) (b)}
  The system described in the question has the follow probabilities for degradation and repair costs
  \begin{center}
    \begin{tabular}{ccc}
      $p(s'|s)$&=&\begin{tabular}{c|ccc}
      $s$\textbackslash $s'$&1&2&3\\\hline
      1&1/2&1/4&1/4\\
      2&ND&1/4&3/4\\
      3&ND&ND&1
      \end{tabular}\\\\
      $c_r(s,s')$&=&\begin{tabular}{c|ccc}
      $s'$\textbackslash $s$&1&2&3\\\hline
      1&ND&1&4\\
      2&ND&ND&2\\
      3&ND&ND&ND
    \end{tabular}
  \end{tabular}
  \end{center}
  where ``ND'' denotes ``Not Defined'', this is due to the actions required for these values to be needed are not admissible in these states.
  \par From the markov decision problem formulated in \texttt{4) (a)} and the given conditions, we can state the following properties of the system
  \[\begin{array}{rcl}
    T&=&\{0,1\}\\
    S&=&\{1,2,3\}\\
    A&=&\{0,1,2\}\\
    A(s)&=&\{0,\dots,s-1\}
  \end{array}\]
  The transition probabilities $p_t(s'|s,a)$ are defined in the tables below, separated by what action $a$ is taken.
  \begin{center}
    \begin{tabular}{ccc}
      $p_t(s'|s,0)$&=&\begin{tabular}{c|ccc}
      $s$\textbackslash $s'$&1&2&3\\\hline
      1&1/2&1/4&1/4\\
      2&0&1/4&3/4\\
      3&0&0&1
      \end{tabular}\\\\
      $p_t(s'|s,1)$&=&\begin{tabular}{c|ccc}
      $s$\textbackslash $s'$&1&2&3\\\hline
      1&NA&NA&NA\\
      2&1/2&1/4&1/4\\
      3&1/2&1/4&1/4
      \end{tabular}\\\\
      $p_t(s'|s,2)$&=&\begin{tabular}{c|ccc}
      $s$\textbackslash $s'$&1&2&3\\\hline
      1&NA&NA&NA\\
      2&NA&NA&NA\\
      3&0&1/4&3/4
      \end{tabular}\\\\
    \end{tabular}
  \end{center}
  where ``NA'' denotes that action $a$ is not an admissible action for that state $s$ (ie $a\not\in A(s)$).
  \par The terminal cost value is $r_2(s)=0$. The cost function values $r_t(s,a)$ are given in the table below and are the same for all $t\in\{0,1\}$.
  \begin{center}
    $r_t(s,a)=$
    \begin{tabular}{c|ccc}
      $s$\textbackslash $a$&0&1&2\\\hline
      1&$c_o(1)$&NA&NA\\
      2&$c_o(2)$&$c_o(2)+c_r(2,1)$&NA\\
      3&$c_o(3)$&$c_o(3)+c_r(3,1)$&$c_o(3)+c_r(3,2)$\\
    \end{tabular}
    =
    \begin{tabular}{c|ccc}
      $s$\textbackslash $a$&0&1&2\\\hline
      1&-1&NA&NA\\
      2&-2&-3&NA\\
      3&-5&-9&-7
    \end{tabular}
  \end{center}
  To find the optimal policy $\pi^*$ we use the \textit{dynamic programming algorithm} which is defined as
  \[\begin{array}{rrl}
    w_t^*(s,a)&:=&r_t(s,a)+\sum_{s'\in S}u_{t+1}^*(s')p_t(s'|s,a)\\
    u_t^*(s)&=&\max_{a\in A(s)}(w_t^*)\\
    d_t^*(s)&=&\argmax_{a\in A(s)}(w_t^*)
  \end{array}\]
  where $u_2^*(s):=r_2(s)=0\ \forall\ s\in S$. Specifically, we need to determine $u_t^*(s),\ d_t^*(s)$ for all states $s$ in each time-period $t\in\{1,0\}$.
  \begin{itemize}
    \item \textit{Time-Period} $t=1$.
    \par In this time-period
    \[\begin{array}{rcl}
      w_1^*(s,a)&=&r_1(s,a)+\sum_{s'\in\{1,2,3\}}u_2^*p_t(s'|s,a)\\
                &=&r_1(s,a)\text{ since }u_{2}^*(s)=0\ \forall\ s\in S
    \end{array}\]
    This gives the following table of values for $w_1^*(s,a)$
    \begin{center}
      $w_1^*(s,a)=$
      \begin{tabular}{c|ccc}
        $s$\textbackslash $a$&0&1&2\\\hline
        1&-1&NA&NA\\
        2&-2&-3&NA\\
        3&-5&-9&-7
      \end{tabular}
    \end{center}
    In all cases taking action $a=0$ (ie not repairing) yields the best results. This is summarised in the following table of results for $u_1^*(s),d_1^*(s)$.
    \begin{center}
      \begin{tabular}{c|c|c}
        $s$&$u_1^*(s)$&$d_1^*(s)$\\\hline
        1&-1&0\\
        2&-2&0\\
        3&-5&0
      \end{tabular}
    \end{center}

    \item \textit{Time-Period} $t=0$.
    \par In this time-period
    \[\begin{array}{rcl}
      w_0^*(s,a)&=&r_0(s,a)+\sum_{s'\in\{1,2,3\}}u_1^*p_t(s'|s,a)\\
                &=&r_0(s,a)-p_t(1|s,a)-2p_t(2|s,a)-5p_t(3|s,a)
    \end{array}\]
    This gives the following table of values for $w_0^*(s,a)$
    \begin{adjustwidth}{-2cm}{-2cm}
        $w_0^*(s,a)=$
        \begin{tabular}{c|ccc}
          $s$\textbackslash $a$&0&1&2\\\hline
          1&-1-1/2-2/4-5/4&NA&NA\\
          2&-2+0-2/4-15/4&-3-1/2-2/4-5/4&NA\\
          3&-5+0+0-5&-9-1/2-2/4-5/4&-7+0-2/4-15/4
        \end{tabular}
        =
        \begin{tabular}{c|ccc}
          $s$\textbackslash $a$&0&1&2\\\hline
          1&-13/4&NA&NA\\
          2&-25/4&-21/4&NA\\
          3&-10&-45/4&-45/4
        \end{tabular}
    \end{adjustwidth}
    This time the optimal action is different for different states. I summarise the optimal actions in the table below
    \begin{center}
      \begin{tabular}{c|c|c}
        $s$&$u_1^*(s)$&$d_1^*(s)$\\\hline
        1&-13/4&0\\
        2&-21/4&1\\
        3&-10&0
      \end{tabular}
    \end{center}
  \end{itemize}
  The optimal strategy $\pi^*$ is
  \[ \pi^*:=(d_1^*(s),d_2^*(s)) \]
  and the optimal value function is
  \[
     u_0^*(s)=\begin{cases}
                -13/4&\text{if }s=0\\
                -21/4&\text{if }s=1\\
                -10&\text{if }s=2
              \end{cases}
  \]

\end{answer}

\end{document}
